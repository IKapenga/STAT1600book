\begin{exsol@exercise}{15.-4}
    \begin{center}
\begin{flushleft}\textbf{ \large \hfill Workshop 2A, Submitted by: }\end{flushleft}

\fbox{\parbox{14cm}{
\vspace{4mm}
Name: \underline{\phantom{xxxxxxxxxxxxxxxxxxxxxxxx}} Signature: \underline{\phantom{xxxxxxxxxxxxxxxxxxxxxxxx}}

\vspace{4mm}
Name: \underline{\phantom{xxxxxxxxxxxxxxxxxxxxxxxx}} Signature: \underline{\phantom{xxxxxxxxxxxxxxxxxxxxxxxx}}

\vspace{4mm}
Name: \underline{\phantom{xxxxxxxxxxxxxxxxxxxxxxxx}} Signature: \underline{\phantom{xxxxxxxxxxxxxxxxxxxxxxxx}}

\vspace{4mm}
Name: \underline{\phantom{xxxxxxxxxxxxxxxxxxxxxxxx}} Signature: \underline{\phantom{xxxxxxxxxxxxxxxxxxxxxxxx}}
 }}
\end{center}

The average test grades of 19 students are as follows (on a scale from 0 to 100, with 100 being the highest score):

91 (A), 95 (A), 96 (A), 82 (B), 94 (A), 75 (C), 91 (A), 79 (C), 92 (A), 100 (A), 89 (B), 93(A), 91 (A), 86 (B), 93 (A), 72 (C), 74 (C), 93 (A), 90 (A)

\begin{table}[ht]
\centering
\caption{Distribution of Grades}
\begin{tabular}{@{} lcc @{}} \hline
%  & \multicolumn{5}{c}{Class } \\ \hline
grades  & frequency & relative frequency \\ \hline
A  & & \\
B  & & \\
C  & & \\ \hline
\end{tabular}
\end{table}

    Using the data from Table 15.3,

\begin{enumerate}
  \item What type of variable is `type of infection' numerical or categorical?
  \item	We can further describe this variable as \underline{\phantom{xxxxxxxxxxxxxxx}}.
  \item	Construct a relative frequency table for the number of infections (show percentages to two places past the decimal).
  \item	Complete a bar chart for the relative frequency table above complete with title and axis Labels.
  \item	What other type of graph can we use for this variable type?
\end{enumerate}
\end{exsol@exercise}
