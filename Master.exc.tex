\begin{exsol@exercise}{14.7-3}
Moviegoers are monitored for their level of anxiety while watching a new horror movie billed to be the “scariest movie of all time.” The intensity of a scene in the movie and anxiety are both measured on numerical scales of 0 to 100. A producer for the movie finds that the correlation coefficient between intensity and anxiety is 0.63, the standard deviation of intensity is 30, and the standard deviation of anxiety is 35.

\begin{enumerate}
  \item What is the slope of the regression equation that predicts anxiety based on intensity?
  \item	How do you properly interpret the slope calculated in part a?
  \item	If the intercept is 15, then what is the regression equation?
  \item	What is the predicted value of anxiety for a scene measured at 90 intensity \\ units?
  \item	If a moviegoer experiences anxiety measured at 82 units during a scene measured at 90 intensity units, then what is the residual for that moviegoer?
  \item	How do you properly interpret the residual in part 5?
\end{enumerate}



\end{exsol@exercise}
