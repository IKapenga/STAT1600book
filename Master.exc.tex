\begin{exsol@exercise}{15.-6}
    \begin{center}
\begin{flushleft}\textbf{\large \hfill Workshop 2C, Submitted by: }\end{flushleft}

\fbox{\parbox{14cm}{
\vspace{4mm}
Name: \underline{\phantom{xxxxxxxxxxxxxxxxxxxxxxxx}} Signature: \underline{\phantom{xxxxxxxxxxxxxxxxxxxxxxxx}}

\vspace{4mm}
Name: \underline{\phantom{xxxxxxxxxxxxxxxxxxxxxxxx}} Signature: \underline{\phantom{xxxxxxxxxxxxxxxxxxxxxxxx}}

\vspace{4mm}
Name: \underline{\phantom{xxxxxxxxxxxxxxxxxxxxxxxx}} Signature: \underline{\phantom{xxxxxxxxxxxxxxxxxxxxxxxx}}

\vspace{4mm}
Name: \underline{\phantom{xxxxxxxxxxxxxxxxxxxxxxxx}} Signature: \underline{\phantom{xxxxxxxxxxxxxxxxxxxxxxxx}}
 }}
\end{center}

The numbers of absences for 20 Stat 1600 students are shown below (on a scale from 0 to 30).

Sample:	10, 0, 1, 3, 15, 6, 2, 1, 0, 21, 25, 11, 9, 7, 4, 5, 12, 28, 17, 8

Sorted:	0, 0, 1, 1, 2, 3, 4, 5, 6, 7, 8, 9, 10, 11, 12, 15, 17, 21, 25, 28


\begin{table}[ht]
\centering
\caption{Distribution of Absences}
\begin{tabular}{@{} lcc @{}} \hline
%  & \multicolumn{5}{c}{Class } \\ \hline
Number of absences & Frequency & Relative frequency \\
0-9 & & \\
10-19 & & \\
20-29 & & \\ \hline
\end{tabular}
\end{table}

\begin{enumerate}
  \item	Fill in the frequency and relative frequency above.
  \item	Complete a stem and leaf plot with the stem representing the 10's place and using 0-2.
  \item	Draw a histogram for the absences using the intervals in the relative frequency table above.
  \item	What percentage of students had number of absences less than 20?
\end{enumerate}

\end{exsol@exercise}
