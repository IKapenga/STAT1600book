\begin{exsol@exercise}{16.-12}
    \begin{center}
\begin{flushleft}\textbf{\large \hfill Workshop 15C, Submitted by: }\end{flushleft}

\fbox{\parbox{14cm}{
\vspace{4mm}
Name: \underline{\phantom{xxxxxxxxxxxxxxxxxxxxxxxx}} Signature: \underline{\phantom{xxxxxxxxxxxxxxxxxxxxxxxx}}

\vspace{4mm}
Name: \underline{\phantom{xxxxxxxxxxxxxxxxxxxxxxxx}} Signature: \underline{\phantom{xxxxxxxxxxxxxxxxxxxxxxxx}}

\vspace{4mm}
Name: \underline{\phantom{xxxxxxxxxxxxxxxxxxxxxxxx}} Signature: \underline{\phantom{xxxxxxxxxxxxxxxxxxxxxxxx}}

\vspace{4mm}
Name: \underline{\phantom{xxxxxxxxxxxxxxxxxxxxxxxx}} Signature: \underline{\phantom{xxxxxxxxxxxxxxxxxxxxxxxx}}
 }}
\end{center}

Frequency of playing violent video games has at times been linked with and seen as a possible cause for those committing violent acts.  Much debate still looms over whether playing violent video games can lead to a desensitization of violence and a lack of compassion for others.

One article that explores these ideas is titled, The Impact of Degree of Exposure to Violent Video Games, Family Background, and Other Factors on Youth Violence, by the authors Whitney Decamp and Christopher Ferguson.  The authors respond to the concern that violence in video games might contribute to youth violence.  There is still no consensus on whether violent video games may be contributing to violence in youth.  The authors sought to study this issue in an ethnically diverse sample of youth in eighth ($n = 5133$) and eleventh grade ($n = 3886$).  Questionnaire surveys were given to the youth to assess a link.  Independent or predictor variables included the playing of violent video games, and an inclination towards violent video games.  Dependent variables included questions pertaining to hitting with the intention to hurt another and taking part in a fight.

Mixed results were reported.  The authors summarized that violent video game play was not significant in 5 of the models used.  There was, however, a positive correlation between games and violent behavior for the models without controls.   When the authors controlled for the propensity and context, however, this suggested the association to be more spurious (not being what it claims to be).  What other factors may be impacting this outcome?

\begin{enumerate}
  \item What possible confounders can you think of for a study of this nature?
  \item Using one confounder from your answer(s) in (1), draw a pathway graph depicting the possible relationships between confounder, possible cause and outcome.
\end{enumerate}
\end{exsol@exercise}
